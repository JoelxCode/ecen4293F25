\documentclass[11pt]{article}
\usepackage[a4paper,margin=1in]{geometry}
\usepackage[T1]{fontenc}
\usepackage{lmodern}
\usepackage{hyperref}
\usepackage{amssymb}
\usepackage{hyperref}
\usepackage{enumitem}
\usepackage{pifont}
\usepackage{titlesec}
\usepackage{listings}
\usepackage{xcolor}
\usepackage{cite}       % For compressed citations like [1], [2]
\usepackage{graphicx}   % For figures
\usepackage{amsmath}    % For math formatting

\hypersetup{
  colorlinks=true,
  linkcolor=blue,
  urlcolor=blue,
  citecolor=blue
}

\titleformat{\section}{\normalfont\Large\bfseries}{}{0pt}{}
\titleformat{\subsection}{\normalfont\large\bfseries}{}{0pt}{}

\lstdefinestyle{cmd}{
  basicstyle=\ttfamily\small,
  frame=single,
  breaklines=true,
  showstringspaces=false,
  columns=fullflexible
}

\title{Opening a Jupyter Notebook in Google Colab}
\author{
  James E. Stine \\
  School of Electrical and Computer Engineering \\
  Oklahoma State University \\
  \texttt{jstine@okstate.edu}
}
\date{\today}

\begin{document}
\maketitle

\section*{Overview}
This short guide explains how to open and run an existing Jupyter
Notebook file (\texttt{.ipynb}) using Google Colab. No local
installation of Python or Jupyter is required.

\section{Background on Google Colab}

Google Colaboratory (Colab) is a cloud-based platform provided by
Google that allows users to create, share, and run Jupyter notebooks
directly in a web browser. It removes the need for local installation
of Python, Jupyter, or scientific libraries, making it an accessible
option for beginners, students, and researchers. Since notebooks are
stored and managed through Google Drive, they can be easily shared and
collaboratively edited, similar to Google Docs.

Colab is built on top of the Jupyter Notebook environment, meaning it
supports the same structure of Markdown cells (for text, equations,
and documentation) and code cells (for Python execution). This
integration makes Colab an ideal tool for teaching, reproducible
research, and rapid prototyping.

One of Colab’s most notable features is free access to cloud-based
hardware accelerators such as GPUs (Graphics Processing Units) and
TPUs (Tensor Processing Units). These resources allow users to run
computationally intensive tasks---such as machine learning, deep
learning, and large-scale data analysis---without requiring
specialized local hardware. Users can select the runtime type and
scale up to hardware that would otherwise be costly or unavailable on
personal machines.

Colab also integrates seamlessly with the Python ecosystem, supporting
popular libraries such as NumPy, Pandas, Matplotlib, TensorFlow, and
PyTorch. Additional packages can be installed directly in a notebook
cell using \texttt{pip} commands. This flexibility makes it useful
across a wide range of applications: education, data science,
computational research, and rapid software development.

In summary, Colab extends the capabilities of Jupyter notebooks by:
\begin{itemize}
\item Removing barriers to entry (no installation required)
\item Enabling real-time collaboration via Google Drive.
    \item Providing free or low-cost access to powerful computing resources.
    \item Maintaining compatibility with standard Jupyter notebook workflows.
\end{itemize}

\section{Upload to Google Colab}
\begin{enumerate}[label=Step \arabic*:, leftmargin=*, itemsep=0.6em]
  \item Open Colab: Navigate to
    \href{https://colab.research.google.com}{colab.research.google.com}.    
  \item Sign in: Use your Google account if you're not already signed
    in.    
  \item Upload the notebook: In the welcome dialog, select the Upload
    tab (alongside \emph{Recent}, \emph{Google Drive},
    \emph{GitHub}). Click Choose file, select your \texttt{.ipynb}
    file (e.g., \texttt{sample\_notebook.ipynb}), and press Open.    
  \item Run cells: Once the notebook opens, click the $\blacktriangleright$ 
    button to the left of any code cell to execute it. Execution
    happens on Google's servers.    
\end{enumerate}

\section{(Optional) Save to Google Drive}
To keep a persistent copy, choose File $\rightarrow$ Save a copy in
Drive. This stores a working copy in your Google Drive for future use
without re-uploading.

\section{(Optional) Local Use Instead of Colab}
If you prefer to run the notebook locally:
\begin{enumerate}[label=\alph*), leftmargin=*, itemsep=0.4em]
  \item Install Jupyter (via Anaconda or \texttt{pip}). For \texttt{pip}, run:
\begin{lstlisting}[style=cmd]
pip install notebook jupyterlab
\end{lstlisting}
  \item Launch Jupyter:
\begin{lstlisting}[style=cmd]
jupyter lab
# or
jupyter notebook
\end{lstlisting}
  \item Use your web browser interface to open or upload the \texttt{.ipynb} file.
\end{enumerate}

\section{Notes}
\begin{itemize}[leftmargin=*]
  \item Colab supports CPU by default; for hardware acceleration,
    visit Runtime $\rightarrow$ Change runtime type to select GPU/TPU
    (availability may vary).    
  \item If your notebook uses extra Python libraries, Colab lets you
    install them in a cell, e.g.:   
\begin{lstlisting}[style=cmd]
> pip install numpy matplotlib
\end{lstlisting}
  \item We will cover more on how to add libraries later in the course.
\end{itemize}

\bigskip
\noindent\textit{Tip:} Share notebooks by saving to Google Drive and using standard Drive sharing, or by hosting them in a GitHub repository and opening via Colab's \emph{GitHub} tab.

\section{Related Resources on Jupyter and Colab}

There are several excellent books and resources available to help users get started with
Jupyter notebooks and Google Colab, ranging from introductory guides to advanced data science
applications.

For those new to the field, \emph{Python for Data Science For Dummies (2nd Edition)} 
provides a very approachable introduction, including a dedicated chapter on 
working with Google Colab \cite{muller2019python}. This is a great starting point 
for learners who want to experiment in the cloud without setting up local environments.

A more focused introduction to Jupyter itself can be found in 
\emph{Jupyter Notebook 101} \cite{driscoll2018jupyter}, which offers a clear 
and concise walkthrough of notebook creation and execution. For a deeper dive, 
\emph{Learning Jupyter} \cite{huff2016learning} expands on advanced functionality, 
such as multiple language kernels, interactive widgets, and integrations with 
data pipelines.

For users who prefer the modern JupyterLab interface, \emph{JupyterLab 101} 
\cite{driscoll2020jupyterlab} provides practical instruction on features such as 
workspace layouts, terminal integration, and debugging support.

From a data science perspective, the widely respected 
\emph{Python Data Science Handbook} \cite{vanderplas2016python} is available as 
an open-source Jupyter notebook collection, covering core scientific libraries 
such as NumPy, Pandas, Matplotlib, and Scikit-Learn. This makes it ideal for use 
within both Jupyter and Google Colab environments.

Machine learning practitioners will benefit from 
\emph{Deep Learning for Coders with fastai and PyTorch} \cite{howard2020deep}, 
which is written entirely as executable notebooks designed to be run in Colab 
or similar platforms. Similarly, the open-source textbook 
\emph{Dive into Deep Learning} \cite{zhang2021dive} integrates theory and 
hands-on code, offering a comprehensive and interactive treatment of deep learning 
concepts.

Together, these resources provide a broad spectrum of learning materials, 
from beginner-level introductions to advanced data science and deep learning 
applications, all leveraging the flexibility of Jupyter notebooks and the 
accessibility of Google Colab.

\bibliographystyle{IEEEtran}
\bibliography{ref}  % This looks for ref.bib



\end{document}
